% LaTeX-2e document, 41 pp, pdfTeX, Version 3.1415926-1.40.10

\documentclass{article}
% Packages
\usepackage[T1]{fontenc}
\usepackage[hyphens]{url}
\usepackage[bookmarks]{hyperref}
\hypersetup{
colorlinks,
citecolor=black,
filecolor=black,
linkcolor=green,
urlcolor=green
}
% \usepackage[latin1]{inputenc}
\usepackage{amsmath}
\usepackage{amssymb}
\usepackage{amsthm}
\usepackage{enumerate}
\usepackage{cite}

% listings settings
\usepackage{textcomp} % makes upquote work
\usepackage{listings}
\lstset{
  float,
  title={\ttfamily\small Sage}, % title of boxes
  gobble=3, % adjusts the amount of space on the left margin
  language=Python,
  basicstyle=\ttfamily\footnotesize,
  %numbers=left,
  %numberstyle=\tiny,
  %numbersep=5pt,
  frame=single,
  tabsize=2,
  breaklines=true, % fixes lines that runover the box
  upquote=true, % makes quotes look correct for sage
}

% Definitions
\def\rrr{\mathbb{R}}
\def\zzz{\mathbb{Z}}
\def\fff{\mathbb{F}}
\def\nnn{\mathbb{N}}
\def\Chi{\raisebox{2pt}{$\chi$}}
\newcommand{\q}{\symbol{39}}

\title{Extending GREAT with Beta distribution p-values and spatial
autocorrelation statistics}
\author{Charles Celerier (cceleri), Yifei Men (ymen),\\
Ahmed Bou-Rabee (bourabee), Andrew Stiles (aostiles),\\
Steven Lee (slee2010), and Nicholas Damien McGee (ndmcgee)}

\begin{document}

\maketitle

\begin{abstract}
  This paper presents two extensions to GREAT. The first extension is caculating
  weights for arrows based on their proximity to the center of the given
  regulatory domains for each term. This weighting allows us to calculate a
  p-value from the regularized incomplete beta function $I_x(\alpha, \beta)$
  where $\alpha~=~\text{sum of the weights assigned to all arrows}$ and
  $\beta~=~(\text{the maximum possible sum of weights}) - \alpha$. The second
  extension is adding spatial autocorrelation statistics. We include three of
  these statistics in our project: (1) the Getis-Ord General G global statistic, (2)
  Moran's I global statistic, and (3) Getis-Ord Gi* local statistic.
\end{abstract}

\section{Weight function}\label{sec:weightFunction}
Each term defines a set of transcription start sites on a genome (hg18 or mm9).
Both of our extensions to GREAT require weights to be assigned to each
arrow-TSS pair. Let's define the absolute distance function
$d:\nnn\times\nnn\rightarrow\zzz$ as:
\[
d(A_p,TSS_p) = TSS_p - A_p.
\]
We assign weights to each arrow-TSS pair based on a normal distribution with
mean $\mu$ and standard deviation $\sigma$ within 1Mb of the TSS and 0
otherwise. More formally, we can define a weight function
$W:\nnn\times\nnn\rightarrow[0,1]$ as:
\[
W(A_p,TSS_p) =
  \begin{cases}
    \frac{1}{\sigma\sqrt{2\pi}}e^{-\frac{(d(A_p, TSS_p)-\mu)^2}{2\sigma^2}} &
    \text{if\quad} d(A_p, TSS_p) \leq 10^6 \\
    0 & \text{otherwise}
  \end{cases}
\]
where
\[
\mu=0 \text{\qquad and\qquad} \sigma = \frac{10^6}{3}.
\]
These weights are assigned by using the python script {\tt findDist.py}.

\section{Beta distribution}
Given an assignment of weights for each arrow-TSS pair, we can calculate a
p-value from the regularized incomplete beta function $I_x(\alpha, \beta)$.
We decided to define $\alpha$ and $\beta$ as follows:
\[
\alpha\quad = \sum_{A_p\in\text{arrows}}\sum_{TSS_p\in\text{TSSs}}W(A_p,TSS_p)
\]
\[
\beta\;=\;\#\text{arrows }\cdot\text{max}\{W(A_p,TSS_p) :
A_p\in\text{arrows}\text{ and } TSS_p\in\text{TSSs}\} - \alpha
\]
The value for $x$ is less clear. We will try a few ideas:
\[
x\;=\;\frac{\text{sum of all weights for the queried term}}{\text{sum of all weights for
all terms}}
\]
\[
x\;=\;\frac{\text{size of the regulatory domain}}{\text{size of the genome}}
\]
\[
x\;=\;\frac{\text{size of regulatory domain with weight at least the mean
weight}}{\text{size of the genome}}
\]
The regularized incomplete beta function was already implemented in the GREAT
source code. We will use that code for this statistic.

\section{Spatial autocorrelation statistics}
We would like to implement a type of hot spot analysis for which regulatory
domains had the largest impact on the p-value (binomial or beta) for the markers
on each term. The three spatial autocorrelation statistis we found can be found
\href{http://www.sce.lsu.edu/cego/documents/reviews/geospatial/spatial_autocorrelation.pdf}{here}.

Here are the null hypotheses for the global statistics:
\begin{enumerate}[1.]
  \item Getis-Ord General G: ``there is not spatial clustering of the arrow''
  \item Moran's I: ``there is not spatial clustering of the arrow associated
    with the regulatory domains''
\end{enumerate}
Getis-Ord Gi* is a local statistic that will indicate clustering of arrows with
low and high weights. This could be a useful mechanism for spotting patterns in
the arrows submitted to GREAT.
\end{document}
